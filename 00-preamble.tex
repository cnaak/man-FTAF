%------------------------------------------------------------------------------------------------------------------------------%
%                                                           Packages                                                           %
%------------------------------------------------------------------------------------------------------------------------------%
\usepackage{booktabs}
\usepackage[english]{babel}
\usepackage[squaren,cdot]{SIunits}
\usepackage{amsmath}
\usepackage{amssymb}
\usepackage{amsthm}
\usepackage{pslatex}
\usepackage[lining]{ebgaramond}
\usepackage{indentfirst}
\usepackage{supertabular}
\usepackage[version=4]{mhchem}          % Chemical formulas
\usepackage{listings}                   % Code/Snippets listings
\usepackage{color}                      % Colors and color name
\usepackage{xspace}
%-------------------------------------------------------------------------------------------------------------------------------
\usepackage[hyperindex,breaklinks]{hyperref} % Required for hyperlinks
%------------------------------------------------------------------------------------------------------------------------------%
%                                                           Commands                                                           %
%------------------------------------------------------------------------------------------------------------------------------%
\hypersetup{%
    hidelinks,
    colorlinks,
    breaklinks=true,
    urlcolor=color3,
    citecolor=color1,
    linkcolor=color1,
    bookmarksopen=false,
    pdftitle={Title},
    pdfauthor={Author}
}
%-------------------------------------------------------------------------------------------------------------------------------
\setlength{\abovecaptionskip}{4pt}
\setlength{\columnsep}{5.5mm}
\setlength{\columnseprule}{0.2pt}
\setlength{\fboxrule}{0.4pt} % Width of the border around the abstract
%-------------------------------------------------------------------------------------------------------------------------------
\definecolor{color1}{RGB}{0,0,90} % Color of the article title and sections
\definecolor{color2}{RGB}{0,20,20} % Color of the boxes behind the abstract and headings
\definecolor{color3}{RGB}{0,0,192} % Color of the article title and sections
%-------------------------------------------------------------------------------------------------------------------------------
\newtheorem{cnj}{Conjecture}
%-------------------------------------------------------------------------------------------------------------------------------
\newcommand{\XXX}[1]{\relax}
%-------------------------------------------------------------------------------------------------------------------------------
\lstdefinelanguage[thisDocOnly]{julia}{
    morekeywords={%
        using,import,include,begin,end,immutable,mutable,struct,type,function,if,else,elseif,while,return,new,with,do},
    sensitive=true,
    morecomment=[l]{\#},
    morestring=[b]",
    morestring=[b]',
    morecomment=[s]{"""}{"""},
    morekeywords=[2]{%
        Any,Number,Bool,Complex,Real,Irrational,Rational,Integer,AbstractFloat,AbstractInterval,Interval,Signed,Unsigned,%
        Int8,Int16,Int32,Int64,Int128,BigInt,UInt8,UInt16,UInt32,UInt64,UInt128,Float16,Float32,Float64,Float128,BigFloat,%
        Array,Tuple,NTuple,NamedTuple,Dict,AbstractString,Char,String,Symbol,Type,Union,UnionAll,cpFC,bProp,IGMix,AtMix,%
        RMix,::}
}
%-------------------------------------------------------------------------------------------------------------------------------
\definecolor{mygrey}{rgb}{0.40, 0.40, 0.40}
\lstset{%
    language=[thisDocOnly]julia,
    basicstyle={\tt\scriptsize},
    keywordstyle={\textbf},
    keywordstyle=[2]{\color{blue}\textbf},
    stringstyle={\color{orange}},
    commentstyle={\color{green}},
    mathescape,
    frame=tb,
    numbers=left,
    numberstyle={\tt\tiny},
    numbersep=4pt,
    stepnumber=1,
    showstringspaces=false,
    tabsize=4,
    breaklines=false,
    breakatwhitespace=false,
}
%-------------------------------------------------------------------------------------------------------------------------------
\newcommand{\code}[1]{{\small\texttt{{#1}}}}
%-------------------------------------------------------------------------------------------------------------------------------
\newcommand{\eqdef}{\stackrel{\text{{\tiny def}}}{=}} % Equal by definition
\newcommand{\eqmod}{\stackrel{\text{{\tiny mod}}}{=}} % Equal by modeling
\newcommand{\eqconj}{\stackrel{\text{{\tiny conj}}}{=}} % Equal by conjecture
%-------------------------------------------------------------------------------------------------------------------------------
\newcommand{\bdc}{{\mbox{\scriptsize BDC}}}
\newcommand{\tdc}{{\mbox{\scriptsize TDC}}}
%-------------------------------------------------------------------------------------------------------------------------------
\newcommand{\NF}{\ensuremath{n_{\mbox{\scriptsize f}}}}
\newcommand{\NA}{\ensuremath{n_{\mbox{\scriptsize air}}}}
\newcommand{\NC}{\ensuremath{n_{\mbox{\scriptsize \ce{C}}}}}
\newcommand{\NH}{\ensuremath{n_{\mbox{\scriptsize \ce{H}}}}}
\newcommand{\NO}{\ensuremath{n_{\mbox{\scriptsize \ce{O}}}}}
\newcommand{\NN}{\ensuremath{n_{\mbox{\scriptsize \ce{N}}}}}
\newcommand{\NCO}{\ensuremath{n_{\mbox{\scriptsize \ce{CO}}}}}
\newcommand{\NHH}{\ensuremath{n_{\mbox{\scriptsize \ce{H2}}}}}
\newcommand{\NOO}{\ensuremath{n_{\mbox{\scriptsize \ce{O2}}}}}
\newcommand{\NNN}{\ensuremath{n_{\mbox{\scriptsize \ce{N2}}}}}
\newcommand{\NCOO}{\ensuremath{n_{\mbox{\scriptsize \ce{CO2}}}}}
\newcommand{\NHHO}{\ensuremath{n_{\mbox{\scriptsize \ce{H2O}}}}}
%-------------------------------------------------------------------------------------------------------------------------------
\newcommand{\MXAIR}{\mathbb{M}_{\mbox{\scriptsize\tt a}}}
\newcommand{\MXFUE}{\mathbb{M}_{\mbox{\scriptsize\tt f}}}
\newcommand{\MXAFU}{\mathbb{M}_{\mbox{\scriptsize\tt af}}}
\newcommand{\MXPRO}{\mathbb{M}_{\mbox{\scriptsize\tt pr}}}
\newcommand{\MXREA}{\mathbb{M}_{\mbox{\scriptsize\tt re}}}
%------------------------------------------------------------------------------------------------------------------------------%
%                                                           Metadata                                                           %
%------------------------------------------------------------------------------------------------------------------------------%
\makeatletter
\immediate\write18{datelog > \jobname.info}
\makeatother
%-------------------------------------------------------------------------------------------------------------------------------
% Journal information
\JournalInfo{engrXiv}
\Archive{Compiled on \input{\jobname.info} -- Version 0
}
% Article title
\PaperTitle{A Finite-Time Air-Fuel Otto Engine Model}
\Authors{%
    C.~Naaktgeboren\textsuperscript{1$\star$} and
    R.~K.~de~O.~e~Silva\textsuperscript{2}
}
\affiliation{%
    \textsuperscript{1}%
    \textit{%
        Adjunct Professor.
        Universidade Tecnológica Federal do Paraná (UTFPR), Câmpus Guarapuava.
        Grupo de Pesquisa em Ciências Térmicas.
}}
\affiliation{%
    \textsuperscript{2}%
    \textit{%
        Graduate Student.
        Universidade Tecnológica Federal do Paraná (UTFPR), PPGEM, Câmpus Curitiba.
}}
\affiliation{%
    \textsuperscript{$\star$}%
    \textbf{Corresponding  author}: NaaktgeborenC$\cdot$PhD\textcircled{a}gmail$\cdot$com
}
\Keywords{%
    Thermodynamics ---
    Finite-Time ---
    Air-Fuel Otto cycle.
}
\newcommand{\keywordname}{Keywords}
\Highlights{%
    Extends the Finite-Time Heat Addition (FTHA) Otto engine model ---
    Provides Julia codes.
    % Highlights go here ---
}
\newcommand{\highlightname}{Highlights}
%-------------------------------------------------------------------------------------------------------------------------------
\Abstract{%
    This work presents an air-fuel model for spark-ignited, reciprocating, internal combustion  engine  cycle,  suitable  to  be
    taught in an advanced undergraduate mechanical engineering (ME) course, that accounts for finite combustion  reaction  time;
    hence, a finite-time air-fuel Otto engine model. The model herein presented  takes  in  many  real  engine  parameters,  and
    expands upon a  recently  published  air-standard  finite-time  heat  addition  model---which  employs  only  pure-substance
    undergraduate ME thermodynamics---by exchanging the external heat addition, peculiar of air-standard models, by  a  pair  of
    chemical reactions of combustion and equilibrium and a reactive gas mixture model. The model is suitable  for  undergraduate
    ME students, as it was developed in the context of a final ME course project, and successfully applied multiple times to  an
    advanced ME discilpine thereafter. The model allows for air/fuel/engine parameter sensibility  studies  to  be  carried  out
    through simulation, and allow students to write their own code and to run their own simulations.
}
%-------------------------------------------------------------------------------------------------------------------------------
