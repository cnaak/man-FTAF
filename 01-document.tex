%------------------------------------------------------------------------------------------------------------------------------%
%                                                          Title Page                                                          %
%------------------------------------------------------------------------------------------------------------------------------%

\thispagestyle{empty} % Removes page numbering from the first page
\flushbottom % Makes all text pages the same height
\maketitle % Print the title and abstract box


%------------------------------------------------------------------------------------------------------------------------------%
%                                                           License                                                            %
%------------------------------------------------------------------------------------------------------------------------------%

\section*{License}

    \scriptsize\noindent%
    \begin{minipage}{\columnwidth}
        \centering\tt
        \includegraphics[height=6.0mm]{cc/by.pdf}\\[0.5\smallskipamount]
        {\scriptsize\url{https://creativecommons.org/licenses/by/4.0/}}
    \end{minipage}
    \normalsize


%------------------------------------------------------------------------------------------------------------------------------%
%                                                      Table Of Contents                                                       %
%------------------------------------------------------------------------------------------------------------------------------%

\tableofcontents


%------------------------------------------------------------------------------------------------------------------------------%
%                                                         Nomenclature                                                         %
%------------------------------------------------------------------------------------------------------------------------------%

\section*{Nomenclature}
\addcontentsline{toc}{section}{Nomenclature}

\newlength{\lencsep}\setlength{\lencsep}{0.8em}
\newlength{\lensymb}\setlength{\lensymb}{3.0em}
\newlength{\lendefn}\setlength{\lendefn}{4.0em}
\newlength{\lenwhat}\setlength{\lenwhat}{\linewidth}
\newlength{\lenWHAT}\setlength{\lenWHAT}{\linewidth}
\addtolength{\lenwhat}{-\lensymb}
\addtolength{\lenwhat}{-\lendefn}
\addtolength{\lenwhat}{-\lencsep}
\addtolength{\lenWHAT}{-\lensymb}

\par\noindent\begin{supertabular}{@{}p{\lensymb}@{}p{\lenwhat}@{\hspace{\lencsep}}p{\lendefn}}
    \multicolumn{3}{@{}l}{\em Latin symbols:} \\
    $n$             & Polytropic exponent, ---,                                     & Eq.~(\ref{eq:n})                  \\
\end{supertabular}

\par\noindent\begin{supertabular}{@{}p{\lensymb}@{}p{\lenwhat}@{\hspace{\lencsep}}p{\lendefn}}
    \multicolumn{3}{@{}l}{\em Greek symbols:} \\
    $\alpha$        & Crank angular position, \rad,                                 & Eq.~(\ref{eq:x})                  \\
\end{supertabular}

\par\noindent\begin{supertabular}{@{}p{\lensymb}@{}p{\lenWHAT}}
    \multicolumn{2}{@{}l}{\em Subscripts:} \\
    ${}_i$                  & process index                                                                             \\
\end{supertabular}
                                                                                                              
%\par\noindent\begin{supertabular}{@{}p{\lensymb}@{}p{\lenWHAT}}                                               
%    \multicolumn{2}{@{}l}{\em Superscripts:} \\                                                               
%    ${}^j$                  & polytropic exponent loop index                                                            \\
%\end{supertabular}


%------------------------------------------------------------------------------------------------------------------------------%
%                                                         Introduction                                                         %
%------------------------------------------------------------------------------------------------------------------------------%

\section{Introduction}

    The treatment of  Otto  cycles  in  undegraduate  engineering  thermodynamics  courses  typically  begins  with  the  ideal,
    air-standard  model,  that  is  characterized   by   isochoric   heat   interactions~\cite{2014-CengelYA+BolesMA-McGrawHill,
    2002-MoranMJ+ShapiroHN-LTC, 1985-WylenG-Wiley, 2015-KroosKA+PotterMC-Cengage}. More advanced courses may include an air-fuel
    version    that    maintains    the    isochoric    release    of    internal    energy    due     to     the     combustion
    reaction~\cite{2012-BrunettiF-Blucher}. Due to the continuous piston motion on traditional crank-rod reciprocating  engines,
    isochoric heating, or, conversely, internal energy release  from  combustion,  represent  \emph{instantaneous}  process  for
    engines running at finite speed~\cite{2017-NaaktgeborenC-IntJMechEngEduc}.

    In one hand, by acccounting for air-fuel reactive mixtures, models can more precisely capture variations in mixture  caloric
    properties along  the  compression  and  expansion  processes,  thus  providing  an  extra  degree  of  refinement  in  work
    calculations~\cite{2012-BrunettiF-Blucher}, as well as being  able  to  derive  a  more  direct  relationship  between  work
    production and fuel consumption, if the internal energy release is calculated from the chemical reaction.

    On the other hand, by neglecting the finite rate of reaction or internal energy release with isochoric reaction,  the  model
    ignores key effects of simultaneous heat and work interactions that shape experimental $P-v$ cycle curves away from the ones
    modeled with isochoric heating or reaction~\cite{2013-MartinsJJG-Publindustria}. Therefore, the isochoric heating hypothesis
    can be more idealized than the one that replaces combustion by an external heating---so that  it  may  make  more  sense  to
    abandon the isochoric heating hypothesis before introducing combustion as to yield progressive model refinements.

    Doing so has the advantage of remaining at pure-substance thermodynamics while making significant improvements on the  cycle
    model, especially in terms  of  the  variety  of  parameters  that  can  be  easily  introduced.  One  such  model,  labeled
    ``FTHA''\footnote{FTHA    is    the    acronym     of     Finite-Time     Heat     Addition.},     has     been     recently
    proposed~\cite{2017-NaaktgeborenC-IntJMechEngEduc} as a middle step towards the modeling of  air-fuel  Otto  cycles  in  the
    context of mechanical engineering education.



