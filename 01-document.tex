%------------------------------------------------------------------------------------------------------------------------------%
%                                                          Title Page                                                          %
%------------------------------------------------------------------------------------------------------------------------------%

\thispagestyle{empty} % Removes page numbering from the first page
\flushbottom % Makes all text pages the same height
\maketitle % Print the title and abstract box


%------------------------------------------------------------------------------------------------------------------------------%
%                                                           License                                                            %
%------------------------------------------------------------------------------------------------------------------------------%

\section*{License}

    \scriptsize\noindent%
    \begin{minipage}{\columnwidth}
        \centering\tt
        \includegraphics[height=6.0mm]{cc/by.pdf}\\[0.5\smallskipamount]
        {\scriptsize\url{https://creativecommons.org/licenses/by/4.0/}}
    \end{minipage}
    \normalsize


%------------------------------------------------------------------------------------------------------------------------------%
%                                                      Table Of Contents                                                       %
%------------------------------------------------------------------------------------------------------------------------------%

\tableofcontents


%------------------------------------------------------------------------------------------------------------------------------%
%                                                         Nomenclature                                                         %
%------------------------------------------------------------------------------------------------------------------------------%

\section*{Nomenclature}
\addcontentsline{toc}{section}{Nomenclature}

\newlength{\lencsep}\setlength{\lencsep}{0.8em}
\newlength{\lensymb}\setlength{\lensymb}{3.0em}
\newlength{\lendefn}\setlength{\lendefn}{4.5em}
\newlength{\lenwhat}\setlength{\lenwhat}{\linewidth}
\newlength{\lenWHAT}\setlength{\lenWHAT}{\linewidth}
\addtolength{\lenwhat}{-\lensymb}
\addtolength{\lenwhat}{-\lendefn}
\addtolength{\lenwhat}{-\lencsep}
\addtolength{\lenWHAT}{-\lensymb}

\par\noindent\begin{supertabular}{@{}p{\lensymb}@{}p{\lenwhat}@{\hspace{\lencsep}}p{\lendefn}}
    \multicolumn{3}{@{}l}{\em Latin symbols:} \\
    $D$             & Piston diameter, \meter,                                      & Eq.~(\ref{eq:S})                  \\
    $k$             & Reactive mixture component index, ---,                        & Sec.~\ref{sec:model.reactm}       \\
    $L$             & Connecting rod length, \meter,                                & Eq.~(\ref{eq:x})                  \\
    $m$             & Mass, \kilogram,                                              & Eq.~(\ref{eq:mm})                 \\
    $\mathrm{mf}$   & Mass fraction, ---,                                           & Eq.~(\ref{eq:mf})                 \\
    $n$             & Chemical amount, \kilo\mole,                                  & Eq.~(\ref{eq:nm})                 \\
    $n$             & Polytropic exponent, ---,                                     & Eq.~(\ref{eq:n})                  \\
    $N$             & Engine's rotation, RPM,                                       & Sec.~\ref{sec:model.engine}       \\
    $p$             & Reactive mixture component count, ---,                        & Sec.~\ref{sec:model.reactm}       \\
    $P$             & Thermodynamic pressure, \kilo\pascal,                         & Eq.~(\ref{eq:?})                  \\
    $r$             & Ratio (engine compression), ---,                              & Eq.~(\ref{eq:r})                  \\
    $R$             & Crank radius, \meter,                                         & Eq.~(\ref{eq:S})                  \\
    $S$             & Piston stroke, \meter,                                        & Eq.~(\ref{eq:S})                  \\
    $t$             & Time, \second,                                                & Sec.~\ref{sec:model.engine}       \\
    $T$             & Thermodynamic temperature, \kelvin,                           & Eq.~(\ref{eq:?})                  \\
    $v$             & Mass-base specific volume, \meter\cubed\per\kilogram,         & Eq.~(\ref{eq:?})                  \\
    $V$             & Volume, \meter\cubed,                                         & Eqs.~(\ref{eq:S}), (\ref{eq:V})   \\
    $\mathrm{y}$    & Molar fraction, ---,                                          & Eq.~(\ref{eq:y})                  \\
    $z$             & Engine's cylinder count, ---,                                 & Sec.~\ref{sec:model.engine}       \\
\end{supertabular}

\par\noindent\begin{supertabular}{@{}p{\lensymb}@{}p{\lenwhat}@{\hspace{\lencsep}}p{\lendefn}}
    \multicolumn{3}{@{}l}{\em Greek symbols:} \\
    $\alpha$        & Crank angular position, \rad,                                 & Eq.~(\ref{eq:x})                  \\
    $\omega$        & Crank angular velocity, \rad\per\second,                      & Sec.~\ref{sec:model.engine}       \\
\end{supertabular}

\par\noindent\begin{supertabular}{@{}p{\lensymb}@{}p{\lenWHAT}}
    \multicolumn{2}{@{}l}{\em Subscripts:} \\
    ${}_{\bdc}$             & Bottom Dead Center, BDC                                                                   \\
    ${}_d$                  & displaced                                                                                 \\
    ${}_i$                  & process index                                                                             \\
    ${}_{LR}$               & relative to $L/R$ (usually a ratio)                                                       \\
    ${}_m$                  & mixture                                                                                   \\
    ${}_{SD}$               & relative to $S/D$ (usually a ratio)                                                       \\
    ${}_{\tdc}$             & Top Dead Center, TDC                                                                      \\
    ${}_u$                  & unitary or unit                                                                           \\
\end{supertabular}
                                                                                                              
%\par\noindent\begin{supertabular}{@{}p{\lensymb}@{}p{\lenWHAT}}                                               
%    \multicolumn{2}{@{}l}{\em Superscripts:} \\                                                               
%    ${}^j$                  & polytropic exponent loop index                                                            \\
%\end{supertabular}


%------------------------------------------------------------------------------------------------------------------------------%
%                                                         Introduction                                                         %
%------------------------------------------------------------------------------------------------------------------------------%

\section{Introduction}

    The treatment of  Otto  cycles  in  undegraduate  engineering  thermodynamics  courses  typically  begins  with  the  ideal,
    air-standard  model,  that  is  characterized   by   isochoric   heat   interactions~\cite{2014-CengelYA+BolesMA-McGrawHill,
    2002-MoranMJ+ShapiroHN-LTC, 1985-WylenG-Wiley, 2015-KroosKA+PotterMC-Cengage}. More advanced courses may include an air-fuel
    version    that    maintains    the    isochoric    release    of    internal    energy    due     to     the     combustion
    reaction~\cite{2012-BrunettiF-Blucher}. Due to the continuous piston motion on traditional crank-rod reciprocating  engines,
    isochoric heating, or, conversely, internal energy release  from  combustion,  represent  \emph{instantaneous}  process  for
    engines running at finite speed~\cite{2017-NaaktgeborenC-IntJMechEngEduc}.

    In one hand, by acccounting for air-fuel reactive mixtures, models can more precisely capture variations in mixture  caloric
    properties along  the  compression  and  expansion  processes,  thus  providing  an  extra  degree  of  refinement  in  work
    calculations~\cite{2012-BrunettiF-Blucher}, as well as being  able  to  derive  a  more  direct  relationship  between  work
    production and fuel consumption, if the internal energy release is calculated from the chemical reaction.

    On the other hand, by neglecting the finite rate of reaction or internal energy release with isochoric reaction,  the  model
    ignores key effects of simultaneous heat and work interactions that shape experimental $P-v$ cycle curves away from the ones
    modeled with isochoric heating or reaction~\cite{2013-MartinsJJG-Publindustria}. Therefore, the isochoric heating hypothesis
    can be more idealized than the one that replaces combustion by an external heating---so that  it  may  make  more  sense  to
    abandon the isochoric heating hypothesis before introducing combustion as to yield progressive model refinements.

    Doing so has the advantage of remaining at pure-substance thermodynamics while making significant improvements on the  cycle
    model, especially in terms  of  the  variety  of  parameters  that  can  be  easily  introduced.  One  such  model,  labeled
    `FTHA'\footnote{FTHA    is     the     acronym     of     Finite-Time     Heat     Addition.},     has     been     recently
    proposed~\cite{2017-NaaktgeborenC-IntJMechEngEduc} as a middle step towards the modeling of  air-fuel  Otto  cycles  in  the
    context of mechanical engineering education.

    In order to incorporate in a single model effects of (i)~finite duration of the combustion  process,  and  of  (ii)~reactive
    mixture properties, this work proposes a finite-time air-fuel Otto engine model, labeled `FTAF'\footnote{FTAF is the acronym
    of Finite-Time Air-Fuel.}, also in the context of mechanical engineering education. Thus, the model herein presented can  be
    thought of as a \emph{merge} between two existing schemes: (i)~a \emph{modified} version of  the  isochoric  air-fuel  model
    presented  by  Brunetti  \emph{et  al.\/}~\cite{2012-BrunettiF-Blucher},  as  to  avoid  being   limited   to   carbon-based
    fuels~\footnote{This shortcomming comes artificially by the way the combustion reaction  is  normalized,  and  unnecessarily
    prevents fuels such as pure  Hydrogen,  \ce{H2},  or  Hydrazine,  \ce{H2N2},  from  being  simulated.},  and  (ii)~the  FTHA
    model~\cite{2017-NaaktgeborenC-IntJMechEngEduc}.

    The model herein proposed keeps undergraduate mechanical engineering education as  its  audience,  and  differs  from  other
    models in the sense that it avoids further complications present in other models, such  as  friction  and/or  heat  transfer
    irreversibilities~\cite{2008-CurtoRissoPL+HernandezAC-JApplPhys, 2002-CatonJA-IntJMechEngEduc},  intake  and  exit  manifold
    flows~\cite{2001-CatonJA-IntJMechEngEduc}, as well as forming a complete, ready to teach unit, rather than a  collection  of
    add-on  models---including  a  finite  interval  heat  release  one---than  can  be  selected  and  taylored  for  a   given
    application~\cite{2013-MartinsJJG-Publindustria}.  Moreover,  the  model  proposed  in  this  work   is   intended   to   be
    computationally implemented by the students, whether in whole or in part, rather than being presented as a  simulation-ready
    tool~\cite{2011-ZuecoJ-IntJMechEngEduc}.


%------------------------------------------------------------------------------------------------------------------------------%
%                                                      Model Formulation                                                       %
%------------------------------------------------------------------------------------------------------------------------------%

\section{Model Formulation}\label{sec:model}

    The proposed finite-time air-fuel model comprises an (i)~engine model, a (ii)~reactive mixture model, a (iii)~combustion and
    chemical equilibrium model, and a  (iv)~cycle  model.  Models  (i)  and  (iv)  herein  presented  closely  follow  the  FTHA
    ones~\cite{2017-NaaktgeborenC-IntJMechEngEduc}, incorporating minor formulation and discretization  changes.  Present  model
    (ii) of the current work expands upon the FTHA one in the sense that a reactive mixture replaces  the  pure  substance;  the
    mixture, however, remains an ideal one with variable specific heats. The model (iii) of the present work, entirely absent in
    the air-standard FTHA, expands upon the scheme  of  Brunetti~\cite{2012-BrunettiF-Blucher}  in  the  energy  balance,  fluid
    formulations, and on residual gas fractions.


    %---------------------------------------------------------------------------------------------------------------------------
    \subsection{Engine model}\label{sec:model.engine}

    The thermodynamic system is considered as the volume comprised  of  the  closed-valves  combustion  chamber  surfaces  of  a
    standard crank-rod reciprocating internal combustion engine (ICE) of displaced volume $V_d$,  \meter\cubed,  $z$  cylinders,
    and $V_{du} = V_d / z$ unit cylinder displacement volume, \meter\cubed. Combustion chamber minimum and maximum  volumes  are
    $V_\tdc$ and $V_\bdc$ for the top and bottom dead centers, respectively.

    The cylinder diameter is $D$, the piston stroke is $S$, the rod length is $L$, and the crank radius is $R$, all  in  \meter.
    Important ratios include the stroke-to-diameter one, $r_{SD} = S/D$, the rod-to-crank one, $r_{LR} = L/R >  1$,  the  engine
    compression ratio $r$, so that%
    %
    \begin{align}
        \label{eq:S}
        S       &= 2 R = \frac{4 V_{du}}{\pi D^{2}}, \\
        \label{eq:Vdu}
        V_{du}  &= V_\bdc - V_\tdc = \frac{\pi SD^2}{4}, \quad\mbox{and} \\
        \label{eq:r}
        r       &= \frac{V_\bdc}{V_\tdc} = 1 + \frac{V_{du}}{V_\tdc}.
    \end{align}

    The crankshaft angular position $\alpha$, in \radian, and engine angular velocity $\omega$, in \radian\per\second,  and  the
    engine RPM $N$, are related  by  $d\alpha/dt  =  \omega  =  2\pi  N/60$,  in  which  $\alpha$  is  an  integer  multiple  of
    $\unit{2\pi}{\radian}$ whenever the piston is at the TDC. Constant-$\omega$ cases  have  $\alpha  =  \omega  t$  with  $t  =
    \unit{0}{\second}$ at an arbitrary TDC. Moreover, the piston position $x$ with respect to the TDC, in \meter,  is  expressed
    as%
    %
    \begin{equation}
        \label{eq:x}
        x(\alpha) = \left|
                \left[\begin{matrix}L & R\end{matrix}\right] \times
                \left[\;\begin{matrix}
                    1 - \sqrt{\smash[b]{1 - (\sin\alpha / r_{LR})^2}} \\
                    1 - \cos\alpha
                \end{matrix}\;\right]
            \right|.
    \end{equation}%

    The thermodynamic system \emph{instantaneous} volume $V$ is therefore given by%
    %
    \begin{equation}
        \label{eq:V}
        V(\alpha) = V_\tdc + \frac{\pi D^2}{4}x(\alpha).
    \end{equation}

    The engine model, being comprised of simple algebric equations, is left for the students to implement.


    %---------------------------------------------------------------------------------------------------------------------------
    \subsection{Reactive mixture model}\label{sec:model.reactm}

    The reactive mixture is comprised of $p$ pure components, refered to by their $k$ index. By hypothesis, the mixture displays
    an ideal $P$-$v$-$T$ behavior. Elementary mixture  definitions,  in  which  the  ``$m$''  subscript  indicates  ``mixture'',
    include:
    %
    \begin{align}
        \label{eq:mf}
        \mathrm{mf}_k   &= \frac{m_k}{m_m},\\
        \label{eq:y}
        \mathrm{y}_k    &= \frac{n_k}{n_m},\\
        \label{eq:mm}
        m_m             &= \sum_{k=1}^p m_k, \quad\mbox{and} \\
        \label{eq:nm}
        n_m             &= \sum_{k=1}^p n_k,
    \end{align}
    %
    \noindent where $\mathrm{mf}$ and $\mathrm{y}$ are mixture component mass fractions and molar fractions, respectively. Terms
    $m$ and $n$ are mass, in \kilogram, and chemical amount, in \kilo\mole, respectively.

