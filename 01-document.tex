% !center 188 | frame -f '\%-\% ' 188
%----------------------------------------------------------------------------------------------------------------------------------------------------------------------------------------------%
%                                                                                          Title Page                                                                                          %
%----------------------------------------------------------------------------------------------------------------------------------------------------------------------------------------------%

\thispagestyle{empty} % Removes page numbering from the first page
\flushbottom % Makes all text pages the same height
\maketitle % Print the title and abstract box


%----------------------------------------------------------------------------------------------------------------------------------------------------------------------------------------------%
%                                                                                           License                                                                                            %
%----------------------------------------------------------------------------------------------------------------------------------------------------------------------------------------------%

\section*{License}

    \scriptsize\noindent%
    \begin{minipage}{\columnwidth}
        \centering\tt
        \includegraphics[height=6.0mm]{cc/by.pdf}\\[0.5\smallskipamount]
        {\scriptsize\url{https://creativecommons.org/licenses/by/4.0/}}
    \end{minipage}
    \normalsize


%----------------------------------------------------------------------------------------------------------------------------------------------------------------------------------------------%
%                                                                                      Table Of Contents                                                                                       %
%----------------------------------------------------------------------------------------------------------------------------------------------------------------------------------------------%

\tableofcontents


%----------------------------------------------------------------------------------------------------------------------------------------------------------------------------------------------%
%                                                                                         Nomenclature                                                                                         %
%----------------------------------------------------------------------------------------------------------------------------------------------------------------------------------------------%

\section*{Nomenclature}
\addcontentsline{toc}{section}{Nomenclature}

% Primary nomenclature lengths
\newlength{\lencsep}\setlength{\lencsep}{0.8em} % column separation width
\newlength{\lensymb}\setlength{\lensymb}{3.5em} % symbol column width
\newlength{\lendefn}\setlength{\lendefn}{4.5em} % definition column width
% Derived nomenclature lengths
\newlength{\lensybb}\setlength{\lensybb}{\lensymb}
\addtolength{\lensybb}{-\lencsep}
\newlength{\lenwhat}\setlength{\lenwhat}{\linewidth}
\addtolength{\lenwhat}{-\lensymb}
\addtolength{\lenwhat}{-\lendefn}
\newlength{\lenwhab}\setlength{\lenwhab}{\lenwhat}
\addtolength{\lenwhab}{-\lencsep}
\newlength{\lenWHAT}\setlength{\lenWHAT}{\linewidth}
\addtolength{\lenWHAT}{-\lensymb}
% Nomenclature commands
\newcommand{\tcs}[1]{\parbox[t]{\lensybb}{\ensuremath {#1}}}    % tabbing column for symbol
\newcommand{\tcw}[1]{\parbox[t]{\lenwhab}{\raggedright{#1}}}    % tabbing column for what (explanation)
\newcommand{\tcW}[1]{\parbox[t]{\lenwhab}{{#1}}}                % tabbing column for what (explanation)
\newcommand{\tcd}[1]{\parbox[t]{\lendefn}{\raggedright{#1}}}    % tabbing column for definition

\begin{tabbing}\hspace*{\lensymb}\=\hspace*{\lenwhat}\=\hspace*{\lendefn}\kill
    {\em Latin symbols:} \\
    \tcs{a}                 \> \tcw{Specific heat model coefficients, (variable),}                                                      \> \tcd{Eq.~(\ref{eq:cp})}                          \\
    \tcs{c}                 \> \tcw{Chemical amount for the \ce{CO}, \kilo\mole,}                                                       \> \tcd{Tab.~\ref{tab:molBal}, Eq.~(\ref{eq:c})}    \\
    \tcs{\bar{c}}           \> \tcw{Molar-base specific heat, \kilo\joule\per\kilo\mole\usk\kelvin,}                                    \> \tcd{Eq.~(\ref{eq:cp})}                          \\
    \tcs{\mathbb{C}}        \> \tcw{Polytropic process constant, \kilo\pascal\meter$^{3n}$,}                                            \> \tcd{Sec.~\ref{sec:model.cyclem}}                \\
    \tcs{D}                 \> \tcw{Piston diameter, \meter,}                                                                           \> \tcd{Eq.~(\ref{eq:S})}                           \\
    \tcs{g}                 \> \tcw{Cumulative heat release fraction function, ---,}                                                    \> \tcd{Eq.~(\ref{def:y})}                          \\
    \tcs{\mathtt{g}}        \> \tcw{Fitting auxiliary function, ---,}                                                                   \> \tcd{Eqs.~(\ref{eq:zeta}), (\ref{eq:zeta.g})}    \\
    \tcs{\bar{h}}           \> \tcw{Molar-base specific enthalpy, \kilo\joule\per\kilo\mole,}                                           \> \tcd{Eq.~(\ref{eq:H.comp})}                      \\
    \tcs{H}                 \> \tcw{Enthalpy, \kilo\joule,}                                                                             \> \tcd{Eq.~(\ref{eq:H.comp})}                      \\
    \tcs{k}                 \> \tcw{Mixture component index, ---,}                                                                      \> \tcd{Sec.~\ref{sec:model.reactm}}                \\
    \tcs{K}                 \> \tcw{Reaction equilibrium constant, ---,}                                                                \> \tcd{Eq.~(\ref{eq:K})}                           \\
    \tcs{L}                 \> \tcw{Connecting rod length, \meter,}                                                                     \> \tcd{Eq.~(\ref{eq:x})}                           \\
    \tcs{m}                 \> \tcw{Mass, \kilogram,}                                                                                   \> \tcd{Eqs.~(\ref{eq:mf}), (\ref{eq:mEoS})}        \\
    \tcs{\mathbb{M}}        \> \tcw{Mixture, ---,}                                                                                      \> \tcd{Eq.~(\ref{eq:MXAIR})--(\ref{eq:MXREA})}     \\
    \tcs{\mathrm{mf}}       \> \tcw{Mass fraction, ---,}                                                                                \> \tcd{Eq.~(\ref{eq:mf})}                          \\
    \tcs{n}                 \> \tcw{Chemical amount, \kilo\mole,}                                                                       \> \tcd{Eq.~(\ref{eq:nm})}                          \\
    \tcs{n}                 \> \tcw{Fuel composition parameter, \kilo\mole,}                                                            \> \tcd{Eq.~(\ref{eq:epsilon})}                     \\
    \tcs{\mathsf{n}}        \> \tcw{Subprocess polytropic exponent, ---,}                                                               \> \tcd{Eqs.~(\ref{eq:Wi}), (\ref{eq:n})}           \\
    \tcs{N}                 \> \tcw{Engine's rotation, RPM,}                                                                            \> \tcd{Sec.~\ref{sec:model.engine}}                \\
    \tcs{p}                 \> \tcw{Mixture component count, ---,}                                                                      \> \tcd{Sec.~\ref{sec:model.reactm}}                \\
    \tcs{P}                 \> \tcw{Thermodynamic pressure, \kilo\pascal,}                                                              \> \tcd{Eq.~(\ref{eq:uf0})}                         \\
    \tcs{Q}                 \> \tcw{Heat interaction to system, \kilo\joule,}                                                           \> \tcd{Eqs.~(\ref{eq:EnBal.1})--(\ref{eq:EnBal.3})}\\
    \tcs{r}                 \> \tcw{Ratio (engine compression), ---,}                                                                   \> \tcd{Eq.~(\ref{eq:r})}                           \\
    \tcs{R}                 \> \tcw{Crank radius, \meter,}                                                                              \> \tcd{Eq.~(\ref{eq:S})}                           \\
    \tcs{\bar{R}}           \> \tcw{The universal gas constant, \kilo\joule\per\kilo\mole\usk\kelvin,}                                  \> \tcd{Sec.~\ref{sec:model.reactm}}                \\
    \tcs{S}                 \> \tcw{Piston stroke, \meter,}                                                                             \> \tcd{Eq.~(\ref{eq:S})}                           \\
    \tcs{t}                 \> \tcw{Time, \second,}                                                                                     \> \tcd{Sec.~\ref{sec:model.engine}}                \\
    \tcs{T}                 \> \tcw{Thermodynamic temperature, \kelvin,}                                                                \> \tcd{Eq.~(\ref{eq:cp})}                          \\
    \tcs{\bar{u}}           \> \tcw{Molar-base specific internal energy, \kilo\joule\per\kilo\mole,}                                    \> \tcd{Eq.~(\ref{eq:U.comp})}                      \\
    \tcs{U}                 \> \tcw{Internal energy, \kilo\joule,}                                                                      \> \tcd{Eq.~(\ref{eq:U.comp})}                      \\
    \tcs{v}                 \> \tcw{Mass-base specific volume, \meter\cubed\per\kilogram,}                                              \> \tcd{Eq.~(\ref{eq:v})}                           \\
    \tcs{\bar{v}}           \> \tcw{Molar-base specific volume, \meter\cubed\per\kilo\mole,}                                            \> \tcd{Sec.~\ref{sec:model.cyclem}}                \\
    \tcs{V}                 \> \tcw{Volume, \meter\cubed,}                                                                              \> \tcd{Eqs.~(\ref{eq:S}), (\ref{eq:V})}            \\
    \tcs{w}                 \> \tcw{Uncertainty of, (variable),}                                                                        \> \tcd{Eq.~(\ref{eq:cp})}                          \\
    \tcs{W}                 \> \tcw{Work interaction to system, \kilo\joule,}                                                           \> \tcd{Eqs.~(\ref{eq:EnBal.1})--(\ref{eq:EnBal.3})}\\
    \tcs{y}                 \> \tcw{Cumulative heat release fraction, ---,}                                                             \> \tcd{Eq.~(\ref{def:y})}                          \\
    \tcs{\mathrm{y}}        \> \tcw{Molar fraction, ---,}                                                                               \> \tcd{Eq.~(\ref{eq:y})}                           \\
    \tcs{z}                 \> \tcw{Engine's cylinder count, ---,}                                                                      \> \tcd{Sec.~\ref{sec:model.engine}}                \\
\end{tabbing}

\begin{tabbing}\hspace*{\lensymb}\=\hspace*{\lenwhat}\=\hspace*{\lendefn}\kill
    {\em Greek symbols:} \\
    \tcs{\alpha}            \> \tcW{Crank angular position, \rad,}                                                                      \> \tcd{Eq.~(\ref{eq:x})}                           \\
    \tcs{\beta}             \> \tcW{Chemical equilibrium solution term, ---,}                                                           \> \tcd{Eqs.~(\ref{eq:c}), (\ref{eq:beta})}         \\
    \tcs{\gamma}            \> \tcW{Chemical equilibrium solution term, ---,}                                                           \> \tcd{Eqs.~(\ref{eq:c}), (\ref{eq:gamma})}        \\
    \tcs{\delta}            \> \tcW{Angular duration of combustion, \rad,}                                                              \> \tcd{Eq.~(\ref{def:y})}                          \\
    \tcs{\Delta}            \> \tcW{Variation operator, ---,}                                                                           \> \tcd{---}                                        \\
    \tcs{\epsilon}          \> \tcW{Amount of fuel per {\kilo\mole} of \ce{O2} that is stoichiometrically oxidized with\-out forming                                                 
                                    Nitrogen oxides, \kilo\mole,}                                                                       \> \tcd{Eq.~(\ref{eq:epsilon})}                     \\
    \tcs{\zeta}             \> \tcW{Gas products residual fraction from one cycle to the next, ---,}                                    \> \tcd{Eq.~(\ref{eq:zeta})}                        \\
    \tcs{\theta}            \> \tcW{Ignition angle, \rad,}                                                                              \> \tcd{Eq.~(\ref{def:y})}                          \\
    \tcs{\phi}              \> \tcW{Actual-to-stoichiometric ratio of fuel-to-air ratios, or the equivalence ratio, ---,}               \> \tcd{Eq.~(\ref{eq:phi})}                         \\
    \tcs{\psi}              \> \tcW{Inert-to-oxidant gas molar ratio for the air model, ---,}                                           \> \tcd{Eq.~(\ref{eq:phi})}                         \\
    \tcs{\omega}            \> \tcW{Crank angular velocity, \rad\per\second,}                                                           \> \tcd{Sec.~\ref{sec:model.engine}}                \\
\end{tabbing}

\begin{tabbing}\hspace*{\lensymb}\=\hspace*{\lenWHAT}\kill
    {\em Subscripts:} \\
    \tcs{{}_0}                                  \> end of admission stroke conditions                                                                                                       \\
    \tcs{{}_{\mbox{\scriptsize\tt a}}}          \> that of air mixture                                                                                                                      \\
    \tcs{{}_{\mbox{\scriptsize\tt af}}}         \> that of air-fuel mixture                                                                                                                 \\
    \tcs{{}_{\mbox{\scriptsize air}}}           \> that of air                                                                                                                              \\
    \tcs{{}_{any}}                              \> of any value                                                                                                                             \\
    \tcs{{}_{\bdc}}                             \> Bottom Dead Center, BDC                                                                                                                  \\
    \tcs{{}_c}                                  \> combustion                                                                                                                               \\
    \tcs{{}_{\mbox{\scriptsize \ce{C}}}}        \> that of atomic Carbon                                                                                                                    \\
    \tcs{{}_{\mbox{\scriptsize \ce{CO}}}}       \> that of Carbon Monoxide molecule                                                                                                         \\
    \tcs{{}_{\mbox{\scriptsize \ce{CO2}}}}      \> that of Carbon Dioxide molecule                                                                                                          \\
    \tcs{{}_d}                                  \> displaced                                                                                                                                \\
    \tcs{{}_{DS}}                               \> relative to the $D/S$ ratio                                                                                                              \\
    \tcs{{}_f}                                  \> of formation                                                                                                                             \\
    \tcs{{}_{\mbox{\scriptsize f}}}             \> that of fuel                                                                                                                             \\
    \tcs{{}_{\mbox{\scriptsize\tt f}}}          \> that of fuel mixture                                                                                                                     \\
    \tcs{{}_{\mbox{\scriptsize \ce{H}}}}        \> that of atomic Hydrogen                                                                                                                  \\
    \tcs{{}_{\mbox{\scriptsize \ce{H2}}}}       \> that of Hydrogen molecule                                                                                                                \\
    \tcs{{}_{\mbox{\scriptsize \ce{H2O}}}}      \> that of Water molecule                                                                                                                   \\
    \tcs{{}_i}                                  \> process index                                                                                                                            \\
    \tcs{{}_{LR}}                               \> relative to the $L/R$ ratio                                                                                                              \\
    \tcs{{}_m}                                  \> mixture                                                                                                                                  \\
    \tcs{{}_{max}}                              \> maximum                                                                                                                                  \\
    \tcs{{}_{min}}                              \> minimum                                                                                                                                  \\
    \tcs{{}_{\mbox{\scriptsize \ce{N}}}}        \> that of atomic Nitrogen                                                                                                                  \\
    \tcs{{}_{\mbox{\scriptsize \ce{N2}}}}       \> that of Nitrogen molecule                                                                                                                \\
    \tcs{{}_{\mbox{\scriptsize \ce{O}}}}        \> that of atomic Oxygen                                                                                                                    \\
    \tcs{{}_{\mbox{\scriptsize \ce{O2}}}}       \> that of Oxygen molecule                                                                                                                  \\
    \tcs{{}_p}                                  \> at constant-pressure                                                                                                                     \\
    \tcs{{}_{\mbox{\scriptsize\tt pr}}}         \> that of combustion products mixture                                                                                                      \\
    \tcs{{}_{\mbox{\scriptsize\tt re}}}         \> that of combustion reagents mixture                                                                                                      \\
    \tcs{{}_{RL}}                               \> relative to the $R/L$ ratio                                                                                                              \\
    \tcs{{}_{SD}}                               \> relative to the $S/D$ ratio                                                                                                              \\
    \tcs{{}_{\tdc}}                             \> Top Dead Center, TDC                                                                                                                     \\
    \tcs{{}_u}                                  \> unitary or unit                                                                                                                          \\
    \tcs{{}_v}                                  \> at constant-specific volume                                                                                                              \\
\end{tabbing}
                                                                                                                      
\begin{tabbing}\hspace*{\lensymb}\=\hspace*{\lenWHAT}\kill
    {\em Superscripts:} \\                                                                       
    \tcs{{}^0}                                  \> at the standard reference state \tcs{(P_0, T_0)}                                                                                         \\
    \tcs{{}^j}                                  \> polytropic exponent converging loop index                                                                                                \\
\end{tabbing}

\begin{tabbing}\hspace*{\lensymb}\=\hspace*{\lenWHAT}\kill
    {\em Chemical species:} \\                                                                       
    \tcs{\ce{C}}                                \> Carbon (atomic)                                                                                                                          \\
    \tcs{\ce{CH4}}                              \> Methane                                                                                                                                  \\
    \tcs{\ce{C12H26}}                           \> Dodecane                                                                                                                                 \\
    \tcs{\ce{CH4O}}                             \> Methanol                                                                                                                                 \\
    \tcs{\ce{C2H6O}}                            \> Ethanol                                                                                                                                  \\
    \tcs{\ce{CO}}                               \> Carbon monoxide                                                                                                                          \\
    \tcs{\ce{CO2}}                              \> Carbon dioxide                                                                                                                           \\
    \tcs{\ce{H}}                                \> Hydrogen (atomic)                                                                                                                        \\
    \tcs{\ce{H2}}                               \> Hydrogen (molecular)                                                                                                                     \\
    \tcs{\ce{H2N2}}                             \> Hydrazine                                                                                                                                \\
    \tcs{\ce{H2O}}                              \> Water                                                                                                                                    \\
    \tcs{\ce{N}}                                \> Nitrogen (atomic)                                                                                                                        \\
    \tcs{\ce{N2}}                               \> Nitrogen (molecular)                                                                                                                     \\
    \tcs{\ce{O}}                                \> Oxygen (atomic)                                                                                                                          \\
    \tcs{\ce{O2}}                               \> Oxygen (molecular)                                                                                                                       \\
\end{tabbing}


%----------------------------------------------------------------------------------------------------------------------------------------------------------------------------------------------%
%                                                                                         Introduction                                                                                         %
%----------------------------------------------------------------------------------------------------------------------------------------------------------------------------------------------%

\section{Introduction}

    The treatment of Otto cycles in undegraduate  engineering  thermodynamics  courses  typically  begins  with  the  ideal,  air-standard  model,  that  is  characterized  by  isochoric  heat
    interactions~\cite{2014-CengelYA+BolesMA-McGrawHill, 2002-MoranMJ+ShapiroHN-LTC, 1985-WylenG-Wiley, 2015-KroosKA+PotterMC-Cengage}. More advanced courses may include  an  air-fuel  version
    that maintains the isochoric release of internal energy due to the combustion  reaction~\cite{2012-BrunettiF-Blucher}.  Due  to  the  continuous  piston  motion  on  traditional  crank-rod
    reciprocating  engines,  isochoric  heating,  or,  conversely,  internal  energy  release  from  combustion,  represent  \emph{instantaneous}  process  for  engines   running   at   finite
    speed~\cite{2017-NaaktgeborenC-IntJMechEngEduc}.

    In one hand, by acccounting for air-fuel reactive mixtures, models can more precisely capture variations in mixture caloric properties along the compression and expansion  processes,  thus
    providing an extra degree of refinement in work calculations~\cite{2012-BrunettiF-Blucher}, as well as being able to derive a more direct relationship  between  work  production  and  fuel
    consumption, if the internal energy release is calculated from the chemical reaction.

    On the other hand, by neglecting the finite rate of reaction or internal energy release with isochoric reaction, the model ignores key effects of simultaneous heat  and  work  interactions
    that shape experimental $P-v$ cycle curves away from the ones modeled with isochoric heating or reaction~\cite{2013-MartinsJJG-Publindustria}. Therefore, the isochoric  heating  hypothesis
    can be more idealized than the one that replaces combustion by an external heating---so that it may make  more  sense  to  abandon  the  isochoric  heating  hypothesis  before  introducing
    combustion as to yield progressive model refinements.

    Doing so has the advantage of remaining at pure-substance thermodynamics while making significant improvements on the cycle model, especially in terms of the variety of parameters that can
    be easily introduced. One such model, labeled `FTHA'\footnote{FTHA is the acronym of Finite-Time Heat Addition.}, has been recently proposed~\cite{2017-NaaktgeborenC-IntJMechEngEduc} as  a
    middle step towards the modeling of air-fuel Otto cycles in the context of mechanical engineering education.

    In order to incorporate in a single model effects of (i)~finite duration of the combustion process, and of (ii)~reactive mixture properties, this work proposes a finite-time air-fuel  Otto
    engine model, labeled `FTAF'\footnote{FTAF is the acronym of Finite-Time Air-Fuel.}, also in the context of mechanical engineering education.  Thus,  the  model  herein  presented  can  be
    thought  of  as  a   \emph{merge}   between   two   existing   schemes:   (i)~a   \emph{modified}   version   of   the   isochoric   air-fuel   model   presented   by   Brunetti   \emph{et
    al.\/}~\cite{2012-BrunettiF-Blucher}, as to avoid being limited to carbon-based fuels~\footnote{This shortcomming comes artificially by the way the combustion reaction is  normalized,  and
    unnecessarily prevents fuels such as pure Hydrogen, \ce{H2}, or Hydrazine, \ce{H2N2}, from being simulated.}, and (ii)~the FTHA model~\cite{2017-NaaktgeborenC-IntJMechEngEduc}.

    The model herein proposed keeps undergraduate mechanical engineering education as its audience, and differs from other models in the sense that it avoids further complications  present  in
    other  models,  such  as  friction  and/or  heat  transfer  irreversibilities~\cite{2008-CurtoRissoPL+HernandezAC-JApplPhys,  2002-CatonJA-IntJMechEngEduc},  intake   and   exit   manifold
    flows~\cite{2001-CatonJA-IntJMechEngEduc}, as well as forming a complete, ready to teach unit, rather than a  collection  of  add-on  models---including  a  finite  interval  heat  release
    one---than can be selected and taylored for a given application~\cite{2013-MartinsJJG-Publindustria}. Moreover,  the  model  proposed  in  this  work  is  intended  to  be  computationally
    implemented by the students, whether in whole or in part, rather than being presented as a simulation-ready tool~\cite{2011-ZuecoJ-IntJMechEngEduc}.


%----------------------------------------------------------------------------------------------------------------------------------------------------------------------------------------------%
%                                                                                      Model Formulation                                                                                       %
%----------------------------------------------------------------------------------------------------------------------------------------------------------------------------------------------%

\section{Model Formulation}\label{sec:model}

    The proposed finite-time air-fuel model comprises an (i)~engine model, a (ii)~reactive mixture model, a (iii)~combustion and chemical equilibrium model, and a (iv)~cycle model. Models  (i)
    and (iv) herein presented closely follow the FTHA ones~\cite{2017-NaaktgeborenC-IntJMechEngEduc}, incorporating minor formulation and discretization changes.  Present  model  (ii)  of  the
    current work expands upon the FTHA one in the sense that a reactive mixture replaces the pure substance; the mixture, however, remains an ideal one with variable specific heats. The  model
    (iii) of the present work, entirely absent in the air-standard FTHA, expands upon the scheme of Brunetti~\cite{2012-BrunettiF-Blucher} in the energy balance,  fluid  formulations,  and  on
    residual gas fractions.


    %-------------------------------------------------------------------------------------------------------------------------------------------------------------------------------------------
    \subsection{Engine model}\label{sec:model.engine}

    The thermodynamic system is considered as the volume comprised of the closed-valves combustion chamber surfaces of a standard crank-rod reciprocating internal combustion  engine  (ICE)  of
    displaced volume $V_d$, \meter\cubed, $z$ cylinders, and $V_{du} = V_d / z$ unit cylinder displacement volume, \meter\cubed. Combustion chamber minimum and maximum volumes are $V_\tdc$ and
    $V_\bdc$ for the top and bottom dead centers, respectively.

    The cylinder diameter is $D$, the piston stroke is $S$, the rod length is $L$, and the crank radius is $R$, all in \meter. Important ratios include the stroke-to-diameter  one,  $r_{SD}  =
    S/D$, the rod-to-crank one, $r_{LR} = L/R > 1$, the engine compression ratio $r$, so that%
    %
    \begin{align}
        \label{eq:S}
        S       &= 2 R = \frac{4 V_{du}}{\pi D^{2}}, \\
        \label{eq:Vdu}
        V_{du}  &= V_\bdc - V_\tdc = \frac{\pi SD^2}{4}, \quad\mbox{and} \\
        \label{eq:r}
        r       &= \frac{V_\bdc}{V_\tdc} = 1 + \frac{V_{du}}{V_\tdc}.
    \end{align}

    It is worth noting that from an implementation point of view, it might be insteresting to define the $r_{SD}$ and the $r_{LR}$ ratios, as well as their  respective  $r_{DS}$  and  $r_{RL}$
    inverses.

    The crankshaft angular position $\alpha$, in \radian, and engine angular velocity $\omega$, in \radian\per\second, and the engine RPM $N$, are related by $d\alpha/dt = \omega = 2\pi N/60$,
    in which $\alpha$ is an integer multiple of $\unit{2\pi}{\radian}$ whenever the piston is at the TDC. Constant-$\omega$ cases have $\alpha = \omega t$ with $t =  \unit{0}{\second}$  at  an
    arbitrary TDC. Moreover, the piston position $x$ with respect to the TDC, in \meter, is expressed as%
    %
    \begin{equation}
        \label{eq:x}
        x(\alpha) = \left|
                \left[\begin{matrix}L & R\end{matrix}\right] \times
                \left[\;\begin{matrix}
                    1 - \sqrt{\smash[b]{1 - (\sin\alpha / r_{LR})^2}} \\
                    1 - \cos\alpha
                \end{matrix}\;\right]
            \right|.
    \end{equation}%

    The thermodynamic system \emph{instantaneous} volume $V$ is therefore given by%
    %
    \begin{equation}
        \label{eq:V}
        V(\alpha) = V_\tdc + \frac{\pi D^2}{4}x(\alpha).
    \end{equation}

    The ignition angle is $\theta$, so that whenever%
    %
    \begin{equation}
        \alpha\!\!\!\mod{2\pi} = \theta,
        \label{eq:theta}
    \end{equation}
    %
    \noindent ignition occurs.

    Moreover, the combustion reaction takes $\Delta t_c$ seconds to complete, after ignition events. For constant-$\omega$ cases, the corresponding angular  combustion  duration  $\delta$,  in
    \radian, is given by%
    %
    \begin{equation}
        \delta = \omega\Delta t_c \qquad\mbox{(const-$\omega$).}
        \label{eq:delta}
    \end{equation}

    The engine model, being comprised of simple algebric equations, is left for the students to implement.


    %-------------------------------------------------------------------------------------------------------------------------------------------------------------------------------------------
    \subsection{Reactive mixture model}\label{sec:model.reactm}

    The reactive mixture is comprised of $p$ pure components, refered to by their $k$ index. By hypothesis, the mixture displays an ideal $P$-$v$-$T$ behavior. Elementary mixture  definitions,
    in which the ``$m$'' subscript indicates ``mixture'', include:%
    %
    \begin{align}
        \label{eq:mf}
        \mathrm{mf}_k   &= \frac{m_k}{m_m}, \\
        \label{eq:y}
        \mathrm{y}_k    &= \frac{n_k}{n_m}, \\
        \label{eq:mm}
        m_m             &= \sum_{k=1}^p m_k, \quad\mbox{and} \\
        \label{eq:nm}
        n_m             &= \sum_{k=1}^p n_k,
    \end{align}
    %
    \noindent where $\mathrm{mf}$ and $\mathrm{y}$ are mixture component mass fractions and molar fractions, respectively. Terms $m$ and $n$ are mass, in \kilogram,  and  chemical  amount,  in
    \kilo\mole, respectively.

    The apparent mixture molar mass $M_m$, in \kilogram\per\kilo\mole, and apparent mixture constant, $R_m$, in \kilo\joule\per\kilogram\usk\kelvin, and mixture (ideal) equation of  state  are
    given by:%
    %
    \begin{align}
        \label{eq:Mm}
        M_m         &= \frac{m_m}{n_m} = \sum_{k=1}^p \mathrm{y}_k M_k, \\
        \label{eq:Rm}
        R_m         &= \frac{\bar{R}}{M_m}, \quad\mbox{and} \\
        \label{eq:mEoS}
        P_mV_m      &= n_m\bar{R}T_m = m_mR_mT_m.
    \end{align}

    Let the components' molar specific heat at constant pressure, $\bar{c}_{p,k}$, in \kilo\joule\per\kilo\mole\usk\kelvin, be modeled as a function of temperature, with  auxiliar  parameters,
    in the form:%
    %
    \begin{equation}
        \bar{c}_{p,k}(T) \eqmod \sum_{i=-4}^4 a_iT^i \pm w_{cp}, \quad T_{min} \leqslant T \leqslant T_{max}.
        \label{eq:cp}
    \end{equation}

    Models like this are frequently associated with a maximum and/or average uncertainties, $w_{cp}$, and validity temperature intervals, $[T_{min},  T_{max}]$;  which  are  either  listed  in
    Thermodynamics texts~\cite{2014-CengelYA+BolesMA-McGrawHill, 1985-WylenG-Wiley} or can be found by a suitable data modeling (regression) procedure.

    By using the ideal gas relation of $\bar{c}_{v,k}(T) = \bar{c}_{p,k}(T) - \bar{R}$, allows one to find component enthalpy and internal energy by a suitable integration.

    Let the enthalpy of a reactive mixture component $k$ portion (of $\unit{n_k}{\kilo\mole}$) at a given temperature $T$ be:%
    %
    \begin{align}
        H_k(T)      &= n_k\bar{h}_k(T) \nonumber\\
                    &= n_k[\bar{h}^0_{f,k} + \bar{h}^0_k(T)] \nonumber\\
                    &= n_k\left[
                        \bar{h}^0_{f,k} + \int_{T_0}^T \bar{c}_{p,k}(T)\,dT
                    \right].
        \label{eq:H.comp}
    \end{align}
    %
    \noindent  where  $\bar{h}^0_{f,k}$,  in  \kilo\joule\per\kilo\mole,  is  the  specific  molar  enthalpy  of  formation  of  substance  $k$  at  the  standard  state  $(P_0,  T_0)   \equiv
    (\unit{1}{\mathrm{atm}}, \unit{298.15}{\kelvin})$; $\bar{h}^0_k(T)$ is the substance $k$ specific molar ideal gas enthalpy of state $(P_{\mathrm{any}}, T)$ with  respect  to  the  standard
    state $(P_0, T_0)$. The bracketed term $\bar{h}_k(T)$ is the substance $k$ specific molar enthalpy with respect to the standard state considering its chemical composition.

    Casting analogous internal energy relations in the form of Eq.~(\ref{eq:H.comp}), one has:%
    %
    \begin{align}
        U_k(T)      &= n_k\bar{u}_k(T) \nonumber\\
                    &= n_k[\bar{u}^0_{f,k} + \bar{u}^0_k(T)] \nonumber\\
                    &= n_k\left[
                        \bar{u}^0_{f,k} + \int_{T_0}^T \bar{c}_{v,k}(T)\,dT
                    \right].
        \label{eq:U.comp}
    \end{align}

    Using $H_k(T) = U_k(T) + PV = U_k(T) + n_k\bar{R}T$, then the definition of $\bar{u}^0_{f,k}$ must be%
    %
    \begin{equation}
        \bar{u}_f^0 \equiv \bar{h}_f^0 - P_0\bar{v}_0 = \bar{h}_f^0 - \bar{R}T_0.
        \label{eq:uf0}
    \end{equation}

    The internal energy expression for the \emph{mixture} in terms of component internal energies is%
    %
    \begin{equation}
        U_m(T) = \sum_{k=1}^p U_k(T) \equiv U^0_{f,m} + U^0_m(T),
        \label{eq:U.mix}
    \end{equation}
    %
    \noindent in which Eqs.~(\ref{eq:U.comp}), and~(\ref{eq:uf0}) can be used as to derive explicit expressions for $U^0_{f,m}$%
    %
    \begin{align}
        U^0_{f,m}   &= \sum_{k=1}^p n_k\bar{u}^0_{f,k} \nonumber\\
                    &= \sum_{k=1}^p n_k\bar{h}_f^0 - n_m\bar{R}T_0 \nonumber\\
                    &= H^0_{f,m} - n_m\bar{R}T_0,
        \label{eq:Uf0.mix}
    \end{align}
    %
    \noindent and for $U^0_m(T)$%
    %
    \begin{align}
        U^0_m(T)    &= \sum_{k=1}^p n_k\bar{u}^0_k(T) \nonumber\\
                    &= \sum_{k=1}^p n_k \int_{T_0}^T \bar{c}_{v,k}(T)\,dT.
        \label{eq:U0.mix}
    \end{align}


    %-------------------------------------------------------------------------------------------------------------------------------------------------------------------------------------------
    \subsection{Combustion and chemical equilibrium model}\label{sec:model.cchemm}

    Following the scheme of Brunetti~\cite{2012-BrunettiF-Blucher}, a general combustion reaction between a fuel model and an air model is to be balanced (solved) in terms of  air,  fuel,  and
    mixture dimensionless parameters. Contrary to the scheme of that reference, in this work, no fuel Carbon content normalization is  going  to  be  performed,  which  unnecessarily  prevents
    non-carbon fuels, such as Hydrogen, \ce{H2}, or Hydrazine, \ce{H2N2}, from being employed in simulations~\cite{2018-SilvaRKO-UTFPR}.

    The air is modeled as a mixture of \ce{O2} and of \ce{N2} gases with a proportion of $\psi$ {\kilo\mole} of \ce{N2} per {\kilo\mole} of \ce{O2}. As the Nitrogen will  be  kept  inert,  all
    inert gases are modeled as \ce{N2}. A typical value of $\psi$ is $79/21 \approx 3.76$.

    Fuel is generically modeled as a \ce{C\NC H\NH O\NO N\NN} mol\-e\-cule, in which \NC, \NH, \NO, and {\NN} are adjustable parameters. Let $\epsilon$ be the amount of fuel  per  {\kilo\mole}
    of \ce{O2} that is stoichiometrically oxidized without forming Nitrogen oxides, therefore,%
    %
    \begin{equation}
        \epsilon^{-1} \equiv \NC + \frac{\NH}{4} - \frac{\NO}{2},
        \label{eq:epsilon}
    \end{equation}
    %
    \noindent so that $\epsilon / (1 + \psi)$ is the stoichiometric fuel-to-air ratio.

    Let $\phi$ be the actual-to-stoichiometric ratio of fuel-to-air ratios, or the equivalence ratio, i.e.,%
    %
    \begin{equation}
        \phi \equiv \frac{\NF/\NA}{\epsilon / (1 + \psi)},
        \label{eq:phi}
    \end{equation}
    %
    \noindent so that $\phi < 1$ models an excess-air mixture, $\phi < 1$ an excess-fuel, and $\phi = 1$ a stoichiometric mixture. Actual amounts of air and fuel, {\NA} and  \NF,  that  occupy
    the closed combustion chamber at the end of the admission stroke, assuming $(P_0, V_0, T_0)$ conditions thereof;  $P_0  \leqslant  P_{\mathrm{atm}}$,  $T_0  \approx  T_{\mathrm{atm}}$  for
    aspirated engines and $V_0 \approx V_\bdc$; are:%
    %
    \begin{align}
        \label{eq:nf}
        \NF     &= \frac{P_0V_0}{\bar{R}T_0}\cdot\frac{\phi\epsilon}{1 + \psi + \phi\epsilon}, \\
        \label{eq:na}
        \NA     &= \frac{P_0V_0}{\bar{R}T_0}\cdot\frac{1 + \psi}{1 + \psi + \phi\epsilon}.
    \end{align}

    The combustion reaction is therefore:%
    %
    \begin{multline}
        \NF\,\mbox{\ce{C\NC H\NH O\NO N\NN}} +\\
        \NA\left(
            \frac{1}{1+\psi}\mbox{\ce{O2}} +
            \frac{\psi}{1+\psi}\mbox{\ce{N2}}
        \right) \ce{->} \\ \shoveright
        {\NCOO\mbox{\ce{CO2}} +
        \NHHO\mbox{\ce{H2O}} +
        \NCO\mbox{\ce{CO}} +} \\
        \NHH\mbox{\ce{H2}} +
        \NOO\mbox{\ce{O2}} +
        \NNN\mbox{\ce{N2}}.
        \label{eq:comb}
    \end{multline}

    By hypothesis, the combustion reaction goes to completion whenever possible, i.e., whenever $\phi \leqslant 1$, so that Carbon and Hydrogen  are  fully  oxidized  into  \ce{CO2}  and  into
    \ce{H2O}, leaving no room for formation of \ce{CO} and of \ce{H2} products. On the other hand, whenever $\phi > 1$, completion hypothesis mean all \ce{O2}  is  used  up  for  complete  and
    partial oxidation of fuel Carbon and Hydrogen, leaving no room for residual \ce{O2} in the products.

    For $\phi \leqslant 1$ under completion hypothesis, Eq.~(\ref{eq:comb}) has enough information as to allow for the molar balance between reagents and products. On the contrary, when  $\phi
    > 1$ there isn't enough information for the molar balance between reagents and products. Closure is obtained by considering the so-called water gas shift (equilibrium) reaction,%
    %
    \begin{equation}
        \mbox{\ce{CO2 + H2 <=> CO + H2O}},
        \label{eq:wgsr}
    \end{equation}
    %
    \noindent and using the equilibrium reaction constant at a given temperature,%
    %
    \begin{equation}
        K(T) = \frac{\NHHO\NCO}{\NCOO\NHH},
        \label{eq:K}
    \end{equation}
    %
    \noindent one gathers enough equations as to perform the molar balance between products and reagents. Brunetti~\cite{2012-BrunettiF-Blucher} employs $K(\unit{1740}{\kelvin})  =  3.5$  with
    the auxiliar variable $c\equiv\NCO$. Using the same method, Table~\ref{tab:molBal} lists balanced reaction product coefficients.

    \begin{table*}[ht]
        \centering
        \caption{Molar balance results for the combustion and water gas shift reactions, Eqs.~(\ref{eq:comb}) and~(\ref{eq:wgsr}) in terms of the equivalence ratio $\phi$}
        \vspace{4pt}
        \begin{tabular}{ccc}
            \toprule
            $n_k$   & air-rich, $\phi \leqslant 1$      & fuel-rich, $\phi > 1$     \\
            \midrule
            $\quad$\NCOO$\quad$ & $\NC\NF \;=\; \NC\dfrac{\phi\epsilon}{1+\psi}\NA$
                                & $\NC\NF - c \;=\; \NC\dfrac{\phi\epsilon}{1+\psi}\NA - c$
                                \\[\bigskipamount]
            \NHHO               & $\frac{\NH}{2}\NF \;=\; \dfrac{\NH}{2}\frac{\phi\epsilon}{1+\psi}\NA$
                                & $(\NO-2\NC)\NF + \dfrac{2}{1+\psi}\NA + c$
                                \\[\bigskipamount]
            \NCO                & $0$
                                & $c$
                                \\[\bigskipamount]
            \NHH                & $0$
                                & $\quad$$\dfrac{2(\phi-1)}{\phi\epsilon}\NF - c \;=\; \dfrac{2(\phi-1)}{1+\psi}\NA - c$$\quad$
                                \\[\bigskipamount]
            \NOO                & $\quad$$(1-\phi)\dfrac{\NA}{1+\psi} \;=\; (1-\phi)\dfrac{\NF}{\phi\epsilon}$$\quad$
                                & $0$
                                \\[\bigskipamount]
            \NNN                & $\dfrac{\psi}{1+\psi}\NA + \dfrac{\NN}{2}\NF$
                                & $\dfrac{\psi}{1+\psi}\NA + \dfrac{\NN}{2}\NF$
                                \\
            \bottomrule
        \end{tabular}
        \label{tab:molBal}
    \end{table*}

    Plugging in Table~\ref{tab:molBal} results for $\phi > 1$ onto Eq.~(\ref{eq:K}) yields a quadratic equation for $c$ whose solution may be written as:%
    %
    \begin{equation}
        \label{eq:c}
        \frac{c}{\NF}   = -\beta \pm \sqrt{\beta^2 - \gamma},
    \end{equation}
    %
    \noindent with%
    %
    \begin{equation}
        \label{eq:gamma}
        \gamma          = \dfrac{2\NC(\phi-1)}{\phi\epsilon(K-1)},
    \end{equation}
    %
    \noindent and%
    %
    \begin{equation}
        \label{eq:beta}
        \beta           = \dfrac{\phi\epsilon[(2-K)\NC-\NO] + 2[K(\phi-1)+1]}{2(K-1)\phi\epsilon}.
    \end{equation}

    Let $\zeta$ be the residual fraction of gas products from a previous cycle that remain in the system for a new cycle. Using the relationship given by Silva~\cite{2018-SilvaRKO-UTFPR}  from
    the data available in Heywood~\cite{1988-HeywoodJB-McGrawHill}, i.e.,%
    %
    \begin{align}
        \zeta(P, r) &= 17.807 + 6.423\mathtt{g}(r) \nonumber\\
                    &\,- [0.029 + 0.013\mathtt{g}(r)] P \nonumber\\
                    &\,+ [1.828 + 0.798\mathtt{g}(v)] \times 10^{-5} \times P^2,
        \label{eq:zeta}
    \end{align}
    %
    \noindent with%
    %
    \begin{equation}
        \mathtt{g}(v) = (5.25 - 0.5r) e^{(8.5 - r)},
        \label{eq:zeta.g}
    \end{equation}
    %
    \noindent so that the following mixtures can be defined:%
    %
    \begin{equation}
        \label{eq:MXAIR}
        \MXAIR  =   \NA\left(
                        \frac{1}{1+\psi}\mbox{\ce{O2}} +
                        \frac{\psi}{1+\psi}\mbox{\ce{N2}}
                    \right),
    \end{equation}
    %
    \begin{equation}
        \label{eq:MXFUE}
        \MXFUE  =   \NF\,\mbox{\ce{C\NC H\NH O\NO N\NN}},
    \end{equation}
    %
    \begin{align}
        \label{eq:MXAFU}
        \MXAFU  &=  \NF\,\mbox{\ce{C\NC H\NH O\NO N\NN}} \nonumber\\
                &\,+\NA\left(
                        \frac{1}{1+\psi}\mbox{\ce{O2}} +
                        \frac{\psi}{1+\psi}\mbox{\ce{N2}}
                    \right),
    \end{align}
    %
    \begin{align}
        \label{eq:MXPRO}
        \MXPRO  &=  \NCOO\mbox{\ce{CO2}} +
                    \NHHO\mbox{\ce{H2O}} +
                    \NCO\mbox{\ce{CO}} \nonumber\\
                &\,+\NHH\mbox{\ce{H2}} +
                    \NOO\mbox{\ce{O2}} +
                    \NNN\mbox{\ce{N2}},
    \end{align}
    %
    \noindent and%
    %
    \begin{equation}
        \label{eq:MXREA}
        \MXREA  =   (1 - \zeta)\,\MXAFU + (\zeta)\,\MXPRO,
    \end{equation}
    %
    \noindent with molar coefficients given by Table~\ref{tab:molBal}.

    Therefore the reactive mixture that comprises the thermodynamic system at the beginning of a cycle is $\MXREA$, i.e., the reagents; while at the end  of  the  cycle,  $\MXPRO$,  i.e.,  the
    products.


    %-------------------------------------------------------------------------------------------------------------------------------------------------------------------------------------------
    \subsection{Cycle model}\label{sec:model.cyclem}

    Similarly to the FTHA model~\cite{2017-NaaktgeborenC-IntJMechEngEduc}, the 4-stroke cycle is modeled as a sequence of locally polytropic sub-processes, according to what has been  proposed
    as a general discrete process type on reference~\cite{2020-NaaktgeborenC-Polytropic-engrXiv-rev02}.

    The process discretization of the air-standard FTHA model \emph{separatedly} splits zero-heat, and heating processes into sub-processes of uniform crankshaft angular increments.  This  has
    the advantage of capturing the exact zero-heat to heating process boundaries, and to prescribe different discretization refinements among the  processes,  and  even  to  allow  for  manual
    (coarse) calculations to be done, for instance, in evaluation settings. However, coarse calculations have the disadvantage that the TDC state may not be visited, thus introducing errors in
    work calculations that are likely significant. Evidently the disadvantages fade away whenever sufficiently small crankshaft angular increments are employed in computations.

    In the present model, uniform and sufficiently small crankshaft angular increments are employed, provided that both BDC and TDC states are visited. This means that  sub-process  crankshaft
    angular increments ought to be a rational number involving the numerator $\unit{180}{\degree} = 2 \cdot 2 \cdot 3 \cdot 3 \cdot \unit{5}{\degree}$, or,  conversely,  that  $\Delta\alpha  =
    \unit{180}{\degree} / I$ with $I \in \mathbb{N}^*$, with the simulation starting unphased from the BDC.

    The FTAF concept of cumulative heat release fraction, $y(\alpha)$ is adapted into a cumulative fraction  of  combustion  reaction  completion,  $y(\alpha)$,  with  $0  \leqslant  y(\alpha)
    \leqslant 1$, with $g(\alpha)$ being its functional form---a functional model parameter that can follow literature proposals of  cossenoidal~\cite{2008-CurtoRissoPL+HernandezAC-JApplPhys},
    and of Wiebe~\cite{2013-MartinsJJG-Publindustria} (exponential) forms---so that $g(\theta) = 0$, $g(\theta+\delta) = 1$, with $g(\alpha)$ being monotonic in the  $\theta  \leqslant  \alpha
    \leqslant \delta + \theta$ interval---$\theta$ is the ignition angle, so that combustion starts whenever $\alpha = \theta$, and $\delta = \omega\Delta t_c$ is the angular duration  of  the
    combustion, and $\Delta t_c$ is the time duration of the combustion process---so that%
    %
    \begin{equation}
        y(\alpha) =
        \begin{cases}
            0         & \text{for } \alpha < \theta, \\
            g(\alpha) & \text{for } \theta \leqslant \alpha \leqslant \theta + \delta, \\
            1         & \text{for } \alpha > \theta + \delta.
        \end{cases}
        \label{def:y}
    \end{equation}

    The above definition of $y(\alpha)$ exempts the solution procedure to ``visit'' the exact process boundary states $\alpha = \theta$ and $\alpha = \theta  +  \delta$,  because  $y(\alpha  <
    \theta) = y(\theta) = 0$ and $y(\alpha > \theta + \delta) = y(\theta + \delta) = 1$, and for $\theta \leqslant \alpha \leqslant \theta + \delta$, $y(\alpha) = g(\alpha)$  will  return  the
    exact cumulative fraction of combustion reaction completion, regardless of previously visited states. Therefore, the discrete $\alpha_i$ values are:%
    %
    \begin{align}
        \label{eq:alphai}
        \alpha_i        &= -\pi + (i - 1)\Delta\alpha, \mbox{with $i \in \mathbb{N}^*,$ with} \\
        \label{eq:dAlpha}
        \Delta\alpha    &= \pi / I \;|\; I \in \mathbb{N}^*,
    \end{align}
    %
    \noindent  where  the  $(i  -  1)$  term  makes  Eq.~(\ref{eq:alphai})  suitable  for  implementation  on  $1$-based  index  languages,  such  as  Julia~\cite{2012-BezansonJ+EdelmanA-CoRR,
    2017-BezansonJ+ShahVB-SIAMRev}.

    Owing the discretization of Eq.~(\ref{eq:alphai}), the cycle is comprised by a set of locally polytropic sub-processes $i$, having discrete polytropic exponents $n_i$ that take the  system
    from state-($i$) to state-($i+1$). The $i$-th (sub-)process energy balance, using the historical conventions for the heat, $Q_i$, and work, $W_i$, interactions during the (sub-)process, is
    then%
    %
    \begin{equation}
        Q_i - W_i                                   = U_{m,i+1} - U_{m,i},
        \label{eq:EnBal.1}
    \end{equation}
    %
    \noindent which, by virtue of Eq.~(\ref{eq:U.mix}) can be rewritten as%
    %
    \begin{equation}
        Q_i + (U^0_{f,m,i} - U^0_{f,m,i+1}) - W_i   = U_{0,m,i+1} - U_{0,m,i},
        \label{eq:EnBal.2}
    \end{equation}
    %
    \noindent in which the $U^0_{f,m,i} - U^0_{f,m,i+1}$ term represents the change in formation internal energy of the system in the subprocess $(i)$--$(i+1)$, i.e., a $\Delta{U_{reac,i}}$
    due to the advance of the chemical reaction in the subprocess, therefore, the subprocess energy balance becomes%
    %
    \begin{equation}
        Q_i + \Delta{U_{reac,i}} - W_i              = U_{0,m,i+1} - U_{0,m,i},
        \label{eq:EnBal.3}
    \end{equation}
    %
    \noindent which clearly shows the distinction between heat transfer interactions (for instance, with the engine block), $Q_i$, and internal energy  generation  due  to  chemical  reaction,
    $\Delta{U_{reac,i}}$, in which%
    %
    \begin{align}
        \Delta{U_{reac,i}}  &\equiv U^0_{f,m,i} - U^0_{f,m,i+1}     \qquad\rightharpoondown \\
                            &= H^0_{f,m,i} - \mathrm{n}_{m,i}\bar{R}T_0 \nonumber\\
                            &\,- H^0_{f,m,i+1} + \mathrm{n}_{m,i+1}\bar{R}T_0.
        \label{eq:dUreac}
    \end{align}

    It is worth noting that $-\bar{R}T_0(\mathrm{n}_{m,i} - \mathrm{n}_{m,i+1})$ does not vanish in general because quantity of  matter,  contrary  to  mass,  is  \emph{not}  conserved  during
    chemical reactions.

    The energy balance can be re-written from Eq.~(\ref{eq:EnBal.3}) as to read%
    %
    \begin{equation}
        U_{0,m,i+1} = U_{0,m,i} + Q_i + \Delta{U_{reac,i}} - W_i,
        \label{eq:evol}
    \end{equation}
    %
    \noindent meaning that if its right-hand-side is known---$Q_i$ can be made zero by hypothesis, for simplicity, by assuming an adiabatic system, but later on  improved,  by  accounting  for
    engine block to system heat interactions---then Eq.~(\ref{eq:evol}) can be used to \emph{evolve} the system to the next, $i+1$, state, as the model implementation, shown below, provides  a
    convenient function to invert $U_{0,m,i+1}(T)$ thus providing the value of $T_{i+1}$,  together  with  fact  that  the  system  volume  and  mass,  $V_{i+1}$  and  $m_m$,  are  known  from
    Eqs.~(\ref{eq:V}) and~(\ref{eq:mEoS}) with $(P, V, T) = (P_0, V_0, T_0)$, with an additional state function (system intensive property) given by%
    %
    \begin{equation}
        \label{eq:v}
        v_{m,i} = \frac{V_i}{m_m},
    \end{equation}
    %
    \noindent with $m_m$ remaining \emph{constant} while the system is closed.

    The net work executed by the system during the $i$-th subprocess, $W_i$, is obtained by integration of the $i$-th  locally  polytropic  process,  $P_iV_i^{\mathsf{n}_i}  =  \mathbb{C}_i  =
    P_{i+1}V_{i+1}^{\mathsf{n}_i}$, thus yielding%
    %
    \begin{equation}
        W_i =
        \begin{cases}
            \dfrac{-P_i}{1-\mathsf{n}_i}\left[V_i - V_{i+1}\left(\frac{V_i}{V_{i+1}}\right)^{\mathsf{n}_i}\right],  & \text{for } \mathsf{n}_i \neq 1 \\[\bigskipamount]
            -P_iV_i\ln\dfrac{V_i}{V_{i+1}},                                                                         & \text{for } \mathsf{n}_i = 1.
        \end{cases}
        \label{eq:Wi}
    \end{equation}

    Following the thermodynamic consistency conjecture, adapted from the FTAF model~\cite{2017-NaaktgeborenC-IntJMechEngEduc}, one has:
    %
    \begin{cnj}[(Thermodynamic consistency)]
        For given heat transfer and internal energy generation, $Q_i +  \Delta{U_{reac,i}}$,  there  exists  a  unique  polytropic  exponent  $\mathsf{n}_i$  so  that  the  polytropic  process
        $P_iV_i^{\mathsf{n}_i} = \mathbb{C}_i = P_{i+1}V_{i+1}^{\mathsf{n}_i}$ results in a work interaction $W_i$, given by Eq.~(\ref{eq:Wi}), and in a change of the  system  internal  energy
        $U_{m,i+1} - U_{m,i}$ that is thermodynamically consistent with the $P$-$v$-$T$ state functions at both end states and that also satisfies the energy balance of Eq.~(\ref{eq:evol}).
        \label{cnj:consistency}
    \end{cnj}

    As detailed in the FTAF model~\cite{2017-NaaktgeborenC-IntJMechEngEduc}, the search for the unique, thermodynamically consistent polytropic exponent  $\mathsf{n}_i$  is  a  fast-converging
    iterative process (inner $j$-loop, $j=1$ initially) of

    \begin{enumerate}
        \item Guessing a $j$-th $\mathsf{n}_i^j = k_m(T_i)$ candidate polytropic exponent; then
        \item Calculating $W_i^j$ using Eq.~(\ref{eq:Wi}); then
        \item Solving for $U_{0,m,i+1}^j$ using Eq.~(\ref{eq:evol}), whose inverse gives $T_{i+1}^j$; then
        \item Determining $P_{i+1}^j$ using the mixture EoS, Eq.~(\ref{eq:mEoS}); then
        \item Correcting the polytropic exponent guess by solving for $\mathsf{n}_i^{j+1}$ with Eq.~(\ref{eq:n}); and
        \item Repeating the above steps with incremented $j$, until some quantity variation between consecutive $j$ iterations falls below a  given  threshold,  for  instance,  $\Delta{W_i}  =
            W_i^{j+1} - W_i^j \leqslant \epsilon_W$.
    \end{enumerate}

    Correcting the polytropic exponent guess is done by solving the process for $\mathsf{n}_i$, based on the available $P$-$v$ data:%
    %
    \begin{equation}
        \mathsf{n}_i^{j+1}  = \frac{\displaystyle \log\frac{P_{i+1}^j}{P_i}}{\displaystyle \log\frac{V_i}{V_{i+1}}}
                            = \frac{\displaystyle \log\frac{P_{i+1}^j}{P_i}}{\displaystyle \log\frac{v_i}{v_{i+1}}}.
        \label{eq:n}
    \end{equation}

    Equation~(\ref{eq:n}) has been  written  with  extensive  and  mass-based  intensive  volumes---$V$'s  and  $v$'s,  respectively---since  \emph{constant-mass},  closed  systems  undergoing
    $(i)$--$(i+1)$ processes have equal $V_i / V_{i+1}$ and $v_i / v_{i+1}$ ratios. It is worth noting; however, that mole-based intensive volume ratios, $\bar{v}_i  /  \bar{v}_{i+1}$,  aren't
    necessarily the same due to the possibility of chemical reactions.

    Now the model has closure. Students can be asked to implement the model in whole or in part, although basic thermodynamics requisites can be provided without  loss  of  benefit,  since  it
    comprises pre-requisite knowledge. If, as a natural progression, student have previously learned the FTHA model, they should have no issues implementing this  extended  FTAF  version.  The
    next section illustrates one possible, however very helpful, simple thermodynamic library implementation to be handed to students.


%----------------------------------------------------------------------------------------------------------------------------------------------------------------------------------------------%
%                                                                                     Model Implementation                                                                                     %
%----------------------------------------------------------------------------------------------------------------------------------------------------------------------------------------------%

\section{Model Implementation}

    For the  purpose  of  this  paper,  selected  library  and  user  (simulation)  code  snippets  are  presented  using  the  Julia  programming  language~\cite{2012-BezansonJ+EdelmanA-CoRR,
    2017-BezansonJ+ShahVB-SIAMRev}---a high-level, well-designed, scientific, fast, and open-source language, that seems to be well suited for  engineering  education  and  research.  Not  all
    implementation pieces of the model are provided here, as to encourage student effort and engagement.

    %-------------------------------------------------------------------------------------------------------------------------------------------------------------------------------------------
    \subsection{Reactive mixture model}

    Parts of the model, like the reactive mixture, Eqs.~(\ref{eq:mf})--(\ref{eq:U0.mix}), can be viewed as an independent thermodynamic \emph{library}; as such, its implementation  can  either
    be given to or requested from the students as a separate simulation resource.

    A $\bar{c}_p$ function coefficients, \code{cpFC}, type structure definition that holds the $w_{cp}$ parameter and the $a_i$ coefficients of Eq.~(\ref{eq:cp}) in  fields  named  \code{prec}
    (of precision) and \code{coef}, respectively, is exemplified on Fig.~\ref{fig:code.cpFC}. The $a_i$ coefficients inside \code{coef} are renamed as to ease code inspections.

    \begin{figure*}[ht]
        \centering
        \begin{lstlisting}
struct cpFC # cp Function Coefficients
   prec::AbstractFloat
   coef::NamedTuple{(:ap0,:ap1,:ap2,:ap3,:ap4,:an1,:an2,:an3,:an4),NTuple{9,AbstractFloat}}
end

struct bProp
   form::String         # Chemical formula
   M::AbstractFloat     # Molecular wt, in kg/kmol
   hf0::AbstractFloat   # Standard form. enthalpy, in kJ/kmol
   cpc::cpFC            # Molar cp(T) coefficients
end

FL = Dict(
    :CO => bProp("CO", 28.011, -110530.0, cpFC(0.0046,     (ap0=+6.29933e+01,
      ap1=-1.02420e-02, ap2=+1.84013e-06, ap3=+1.09246e-11, ap4=-3.12160e-14,
      an1=-3.29283e+04, an2=+1.41879e+07, an3=-2.88219e+09, an4=+2.26073e+11))),
    :CO2 => bProp("CO2", 44.010, -393510.0, cpFC(0.0033,   (ap0=+8.54505e+01,
      ap1=-6.49453e-03, ap2=+6.09921e-08, ap3=+4.27120e-10, ap4=-7.13243e-14,
      an1=-3.45881e+04, an2=+1.14977e+07, an3=-2.13732e+09, an4=+1.65231e+11))),
   # ... remaining entries ...
)

# Allowed substances in a mixture
const FLUIDS = Tuple(sort([keys(FL)...]))

# Keys are substances, quantities are in kmol
struct IGMix <: Any
   mix::NamedTuple{FLUIDS,NTuple{length(FLUIDS),Real}}
end

# Keys are elements, quantities are in kmol
struct AtMix <: Any
   EC::NamedTuple{(:H,:C,:N,:O),NTuple{4,Real}} # "EC" stands for: Element Count
end

struct RMix <: Any
   reag::IGMix    # The reagents Ideal Gas Mixture
   prod::IGMix    # The products Ideal Gas Mixture
   reac::Real     # The degree of reaction, in [0..1]
end

function react(r::RMix, amt::Real) # mixture and amount of reaction to advance
   before = mixture(r)
   s = RMix(r.reag, r.prod, r.reac + amt)
   after = mixture(s)
   return s, s.reac - r.reac, Uf0(before) - Uf0(after)
end
        \end{lstlisting}
        \caption{Listing for the \code{cpFC} structure definition code snippet, for storing $\bar{c}_p\,:\,\bar{c}_p(T)$ model coefficients. Data members include the precision \code{prec}  and
            the \code{coef} coefficients as a \code{NamedTuple}}
        \label{fig:code.gasLib}
    \end{figure*}

    Passing an instance of a \code{cpFC} data type to a suitably written function, allows it to evaluate the $\bar{c}_p(T)$ function of Eq.~(\ref{eq:cp}) with or without uncertainty, i.e.,  as
    a plain value or as a value interval, for instance, by  using  the  \code{In\-ter\-val\-Arith\-me\-tic.jl}  Julia  package~\cite{2020-BenetL+SandersDP-IntervalArithmetic.jl}  of  validated
    numerics~\cite{2011-TuckerW-Princeton-Validated, 2009-MooreR-SIAM-Introduction}, which conforms to the IEEE Std 1788-2015~\cite{2015-RevolN+ThompsonT-IEEE-IA}.

    A fluid library, \code{FL}, Fig.~\ref{fig:code.FL}, can be built by storing in a suitable container type \code{bProp},  basic  properties,  Fig.~\ref{fig:code.bProp},  property  regression
    values, as exemplified.

%    \begin{figure*}[h]
%        \centering
%        \begin{lstlisting}
%struct bProp
%   form::String         # Chemical formula
%   M::AbstractFloat     # Molecular wt, in kg/kmol
%   hf0::AbstractFloat   # Standard form. enthalpy, in kJ/kmol
%   cpc::cpFC            # Molar cp(T) coefficients
%end
%FL = Dict(
%    :CO => bProp("CO", 28.011, -110530.0, cpFC(0.0046,     (ap0=+6.29933e+01,
%      ap1=-1.02420e-02, ap2=+1.84013e-06, ap3=+1.09246e-11, ap4=-3.12160e-14,
%      an1=-3.29283e+04, an2=+1.41879e+07, an3=-2.88219e+09, an4=+2.26073e+11))),
%    :CO2 => bProp("CO2", 44.010, -393510.0, cpFC(0.0033,   (ap0=+8.54505e+01,
%      ap1=-6.49453e-03, ap2=+6.09921e-08, ap3=+4.27120e-10, ap4=-7.13243e-14,
%      an1=-3.45881e+04, an2=+1.14977e+07, an3=-2.13732e+09, an4=+1.65231e+11))),
%   # ... remaining entries ...
%)
%        \end{lstlisting}
%        \caption{Listing for the \code{bProp} basic property container for reactive ideal gas and \code{FL} fluid library code  snippet.  \code{bProp}  data  members  include  the  substance's
%            molecular  ``weight'',  \code{M},  formation  enthalpy,  \code{hf0},  and  the  \code{cpFC}  structure  listed  on  Fig.~\ref{fig:code.cpFC}.   \code{FL}   fluid   library   is   a
%            \code{Dict{Symbol,bProp}} instance, in which disctionary keys are \code{Symbol}s and values are of \code{bProp} type. The illustrated coefficients for the \ce{CO} and \ce{CO2} were
%            obtained    by    curve-fitting    into     Eq.~(\ref{eq:cp})     low-pressure     $\bar{c}_p$     values     obtained     with     the     CoolProp     thermophysical     property
%            library~\cite{2014-BellIH+VincentL-IndEngChemR}.}
%        \label{fig:code.bProp}
%    \end{figure*}

    Doing so allows for $M$ and $\bar{h}_f^0$ properties of \ce{CO2} and \ce{CO} ideal gases to be accessed in Julia user code as \code{FL[CO2].M} and as \code{FL[CO].hf0}, for  instance.  The
    above  \code{cpFC}  coefficients  were  obtained  by  least-square  fitting~\cite{2007-PressWH+FlanneryBP-Cambridge}  the  $\bar{c}_p(T)$  data  given  by   the   CoolProp   thermophysical
    library~\cite{2014-BellIH+VincentL-IndEngChemR} in the $\unit{275}{\kelvin} \leqslant T \leqslant \unit{2750}{\kelvin}$ range for the gases at  very  low  pressures  onto  the  \code{cpFC}
    function coefficients using a separate Python~\cite{1995-vanRossumG-CWI} notebook~\cite{2019-Jupyter-www}. The remaining  $M$  and  $\bar{h}_f^0$  data  were  obtained  from  textbook  and
    reference sources~\cite{2013-CengelYA+BolesMA-AMGH, 1985-WylenG-Wiley, 2006-LideDR-CRC} for a range  of  pure  substances:  methane  (\ce{CH4})  through  dodecane  (\ce{C12H26}),  methanol
    (\ce{CH4O}) and ethanol (\ce{C2H6O}), air components (\ce{N2} and \ce{O2}), as well as remaining combustion products (\ce{H2O}, and \ce{H2}).

    All pure substance component properties defined on Eqs.~(\ref{eq:cp})--(\ref{eq:uf0}), as well as pure-substance versions of $P$, $v$, and $T$ functions can be obtained with the  \code{FL}
    fluid library data.

    The implementation of reactive ideal gas mixtures proceeds based on three additional types: an \code{IGMix} ideal gas mixture type and  its  functionality  set,  an  \code{AtMix}  atomized
    mixture type and its functionality set, and an \code{RMix} reactive mixture type and its functionality. The first type definition is illustrated on Fig.~\ref{fig:code.IGMix}.

%    \begin{figure*}[h]
%        \centering
%        \begin{lstlisting}
%# Allowed substances in a mixture
%const FLUIDS = Tuple(sort([keys(FL)...]))
%# Keys are substances, quantities are in kmol
%struct IGMix <: Any
%   mix::NamedTuple{FLUIDS,NTuple{length(FLUIDS),Real}}
%end
%        \end{lstlisting}
%        \caption{Listing for the \code{IGMix} structure code snippet. \code{IGMix} data member, \code{mix}, is a named tuple of each gas's chemical amount in the represented mixture.}
%        \label{fig:code.IGMix}
%    \end{figure*}

    The \code{IGMix} type describes an ideal gas mixture \emph{system}, with defined molar amounts of each  gas;  thus  enabling  calculations  of  the  remaining  quantities  of  interest  of
    Eqs.~(\ref{eq:mfy})--(\ref{eq:mEoS}), and~(\ref{eq:U.mix})--(\ref{eq:U0.mix}).

    Moreover, inverse functions for mixture internal energies and enthalpies can be implemented with adapted versions of the Newton-Raphson method, that take advantage of  the  fact  that  the
    derivatives of these functions can be known exactly through \code{cpFC} data. One such method, \code{TU0}, which is the  inverse  of  the  $U^0_m(T)$  function,  Eq.~(\ref{eq:U0.mix}),  is
    illustrated on fig.~\ref{fig:code.TU0}.

%    \begin{figure*}[h]
%        \centering
%        \begin{lstlisting}
%# Numerical Methods
%function TU0(a::IGMix, tU0::Real; maxIt = 12, $\epsilon$ = 1.0e-6)
%   Ta, Tb, tU0 = (300.0, 2500.0, tU0)
%   Ua, Ub = (U0(a, i) for i in (Ta, Tb))
%   T = [ Ta + (tU0 - Ua) * (Tb - Ta) / (Ub - Ua) ]
%   it = 0
%   $\Delta$U = U0(a, T[end]) - tU0
%   while abs($\Delta$U) >= $\epsilon$ && it < maxIt
%      $\Delta$T = $\Delta$U / Cv(a, T[end])
%      append!(T, T[end] - $\Delta$T)
%      $\Delta$U = U0(a, T[end]) - tU0
%      it += 1
%   end
%   T[end]
%end
%        \end{lstlisting}
%        \caption{Listing for the \code{TU0} function code snippet.}
%        \label{fig:code.TU0}
%    \end{figure*}

    The \code{AtMix} type describes an \emph{atomized} sum of an ideal gas mixture, in terms of \emph{element count}. The purpose of this type  of  mixture  is  to  perform  and/or  to  assert
    element-wise balancing between a reagent \code{IGMix} and a product one. Therefore, the \code{AtMix} functionality is restricted to  basic  algebric  operations  and  logical  comparisons,
    including the \code{isapprox()}, or \code{$\approx$()} function from the \code{Base} Julia module. The \code{AtMix} type definition is shown on Fig.~\ref{fig:code.AtMix}.

%    \begin{figure*}[h]
%        \centering
%        \begin{lstlisting}
%# Keys are elements, quantities are in kmol
%struct AtMix <: Any
%   EC::NamedTuple{(:H,:C,:N,:O),NTuple{4,Real}} # "EC" stands for: Element Count
%end
%        \end{lstlisting}
%        \caption{Listing for the \code{AtMix} structure code snippet.}
%        \label{fig:code.AtMix}
%    \end{figure*}

    It is worth noting that all library substances---including air, fuel, and combustion products---are composed of \ce{H}, \ce{C}, \ce{N}, and of \ce{O} elements only,  so  that  \code{AtMix}
    only needs to keep track of those elements. As an example, by atomizing \unit{1}{\kilo\mole} of ethanol, \ce{C2H6O}, one gets 6\ce{H} + 2\ce{C} + 1\ce{O}.

    Finally, the \code{RMix} type describes a reactive ideal gas mixture in terms of its reagents, \code{reag}; products, \code{prod}; and its degree of reaction, \code{reac}. The type is thus
    organized, so that the corresponding reactive mixture can be examined at any stage of completion of the chemical reaction---a requirement of the finite-time modeling. The  type  definition
    is illustrated on Fig.~\ref{fig:code.RMix}.

%    \begin{figure*}[h]
%        \centering
%        \begin{lstlisting}
%struct RMix <: Any
%   reag::IGMix    # The reagents Ideal Gas Mixture
%   prod::IGMix    # The products Ideal Gas Mixture
%   reac::Real     # The degree of reaction, in [0..1]
%end
%        \end{lstlisting}
%        \caption{Listing for the \code{RMix} structure code snippet.}
%        \label{fig:code.RMix}
%    \end{figure*}

    Type constructors in Julia can be written as to enforce type conventions and invariants. \code{RMix} constructors can ensure the \code{reac} field is never outside of its $[0;  1]$  range.
    Moreover, Julia \code{structs} are immutable by default, ensuring a valid \code{RMix} remains valid.

    High-level functions that take an \code{RMix} instance as argument can be written as to (i)~produce an \code{IGMix} corresponding to the current stage  of  reaction---an  ``interpolation''
    between the \code{reag} and the \code{prod} mixtures; to (ii)~reset, and to (iii)~advance the degree of reaction, i.e., to \code{react} the mixture a defined amount towards  the  formation
    of products. One such function is illustrated of Fig.~\ref{fig:code.react}.

%    \begin{figure*}[h]
%        \centering
%        \begin{lstlisting}
%function react(r::RMix, amt::Real) # mixture and amount of reaction to advance
%   before = mixture(r)
%   s = RMix(r.reag, r.prod, r.reac + amt)
%   after = mixture(s)
%   return s, s.reac - r.reac, Uf0(before) - Uf0(after)
%end
%        \end{lstlisting}
%        \caption{Listing for the \code{react} function code snippet.}
%        \label{fig:code.react}
%    \end{figure*}

    This function returns three values---a 3-Tuple with the new \code{RMix}, the actual amount of reaction (enforcing \code{reac} limits that, for instance, cannot go past  $100\%$),  and  the
    total release of internal energy due to the chemical reaction, $\Delta{U_{reac}}$, Eq.~(\ref{eq:dUreac}).


%----------------------------------------------------------------------------------------------------------------------------------------------------------------------------------------------%
%                                                                                       Validation Cases                                                                                       %
%----------------------------------------------------------------------------------------------------------------------------------------------------------------------------------------------%

\section{Validation Cases}

    This section brings validation results for (i)~the library implementation of the previous section, and (ii)~FTAF model cases.


